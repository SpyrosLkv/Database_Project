\documentclass[14pt]{report}
\usepackage{extsizes}
\usepackage[utf8]{inputenc}
\usepackage[LGR]{fontenc}
\usepackage[greek]{babel}
\usepackage{alphabeta}
\usepackage{graphicx}
\usepackage{enumitem}
\usepackage{xcolor}
\usepackage[colorlinks, linkcolor=teal]{hyperref}


\makeatletter
\renewcommand{\maketitle}{
	\begin{center}
		\includegraphics[width=0.3\textwidth]{./images/Εθνικό_Μετσόβειο_Πολυτεχνείο.svg.png} % Replace with your image file
		\par\vspace{2cm} % Adjust the vertical spacing as desired
		{\LARGE\@title\par}
		\vspace{1.5cm} % Adjust the vertical spacing as desired
		{\large\@date\par}
		\vspace{1.5cm} % Adjust the vertical spacing as desired
		{\large\@author\par}
	\end{center}
}
\makeatother

\title{\textbf{ \fontsize{28}{30}{\textlatin{User manual of semester project of Databases}} \\ Εθνικό Μετσόβειο Πολυτεχνείο }}
\author{Σπυρίδων Λουκοβίτης 03120120 \\ Χρυσόστομος Κοπίτας 03120136
	\\ Αθανάσιος Σπηλιώτης 03120175}
\date{Μάιος-Ιούνιος 2023}
\begin{document}
	\maketitle
	\hypersetup{linkcolor=teal}
	
	
	\chapter*{\Large Περιεχόμενα}
	\hyperlink{Users}{\Large 1)Περιγραφή Χρηστών} \\
	\hyperlink{Homepage}{\Large 2)Αρχική Σελίδα} \\
	\hyperlink{Buttons}{\Large 3)Ανάλυση κουμπιών} \\
	
	\newpage
	\hypertarget{Users}{}
	\chapter*{\Large Είδη Χρηστών}
	\vspace{\baselineskip}
	
	\hyperlink{studentanchor}{\Large 1) Μαθητής/Δάσκαλος} \\
	\hyperlink{operatoranchor}{\Large 2) Χειριστής} \\
	\hyperlink{adminanchor}{\Large 3) Διαχειριστής} \\
	
	\newpage
	\hypertarget{studentanchor}{}
	\textbf{1. Μαθητής/Δάσκαλος/Χρήστης} \\ \\ 
	Ο Χρήστης έχει τα πιο βασικά δικαιώματα και λειτουργίες της βάσης. \\
	Μπορεί να:
	\begin{enumerate}[label=\alph*)]
		\item Δει όλα τα βιβλία που έχουν καταχωρηστεί στη βιβλιοθήκη του με κριτήρια αναζήτησης τίτλο, κατηγορία και συγγραφέα, στα οποία αποτελέσματα να μπορεί να δημιουργήσει αίτημα κράτησης.
		\item Δει τους παλιούς δανεισμούς και τις παλιές του κρατήσεις
		\item Αλλάξει τα στοιχεία του αν είναι εκπαιδευτικός
		\item Αλλάξει τα τηλέφωνά του σε κάθε περίπτωση
		\item Δει την κάρτα του
		\item Δει τις τρέχουσες αιτήσεις, δανεισμούς και κρατήσεις και να διαγράψει αιτήσεις και κρατήσεις
		\item Υποβάλλει αξιολόγηση σε βιβλίο
		\item Δει τα στοιχεία της βιβλιοθήκης του
	\end{enumerate}
	\newpage
	\hypertarget{operatoranchor}{}
	\textbf{2. Χειριστής} \\ \\ 
	Ο Χειριστής είναι ο υπεύθυνος της βιβλιοθήκης και γι' αυτό διαχειρίζεται τις περισσότερες λειτουργίες της. \\
	Μπορεί να:
	\begin{enumerate}[label=\alph*)]
		\item Δει και κάνει ό,τι και ο χρήστης
		\item Δει και εγκρίνει/απορρίψει εκκρεμείς εγγραφές χρηστών
		\item Δει και εγκρίνει/απορρίψει εκκρεμείς κάρτες χρηστών
		\item Διαχειριστεί τους χρήστες της βιβλιοθήκης
		\item Διαχειριστεί τη βιβλιοθήκη
		\item Διαχειριστεί τις αιτήσεις για βιβλία
		\item Ικανοποιήσει κρατήσεις
		\item Περάσει κατευθείαν δανεισμούς
		\item Καταχωρήσει επιστροφή βιβλίου
		\item Δει καθυστερημένες επιστροφές βιβλίων
		\item Δει ενεργούς δανεισμούς
		\item Αλλάξει βιβλίο που υπάρχει ήδη στη βιβλιοθήκη του
		\item Προσθέσει βιβλίο στη βιβλιοθήκη του
		\item Εγκρίνει/Απορρίψει εκκρεμείς αξιολογήσεις βιβλίων
		\item Εκτελέσει τις ερωτήσεις χειριστή της εκφώνησης
	\end{enumerate}
	\newpage
	\hypertarget{adminanchor}{}
	\textbf{3. Διαχεριστής} \\ \\
	Ο Διαχειριστής είναι ο υπεύθυνος όλων των βιβλιοθηκών και γι' αυτό διαχειρίζεται τις πιο σημαντικές λειτουργίες της βάσης. \\
	Μπορεί να:
	\begin{enumerate}[label=\alph*)]
		\item Δει και κάνει ό,τι και ο χρήστης
		\item Προσθέσει και επεξεργαστεί βιβλιοθήκες
		\item Εγκρίνει/Απορρίψει αιτήσεις για χειριστές και διαχειριστές
		\item Φτιάξει \textlatin{backup} της βάσης ολόκληρης
		\item Ανακτήσει την βάση από ένα \textlatin{backup}
		\item Εκτελέσει τις ερωτήσεις διαχειριστή της εκφώνησης
	\end{enumerate}

	\newpage
	\hypertarget{Homepage}{}
	\chapter*{\Large \latintext{Sign In/Sign Up page}}
	Ανοίγοντας την ιστοσελίδα για πρώτη φορά αντικρίζουμε το εξής \latintext{home page} : \\
	\begin{center}
		\includegraphics[width=\textwidth]{./images/index_1.png} % Replace with your image file
		\par\vspace{2cm} % Adjust the vertical spacing as desired
	\end{center}

	\begin{enumerate}
		\newpage
		\item Το \latintext{Home button} θα βρίσκεται κάθε στιγμή πάνω δεξιά όπου και αν είναι για να μπορώ να πάω πίσω στην αρχική σελίδα. \\ \\
		\item Το \latintext{Signup button} υπάρχει για την πρώτη σύνδεσή μας στην βιβλιοθήκη.
		\begin{center}
			\includegraphics[width=\textwidth]{./images/index_3.png}
		\end{center}
		Πατάμε το κουμπί όπως φαίνεται στην εικόνα και καταλήγουμε στην εξής σελίδα: \\
		\includegraphics[width=\textwidth]{./images/signup}
		\par\vspace{2cm}
		Εδώ απαραίτητα πεδία για να καταθέσει ένας χρήστης μια αίτηση είναι: 
		\begin{enumerate}
			\item \latintext{First Name}
			\item \latintext{Last Name}
			\item \latintext{Username}
			\item \latintext{Email address}
			\item \latintext{Password}
		\end{enumerate}
		Ο αριθμός τηλεφώνου και η ημερομηνία γέννησης είναι προαιρετικά πεδία. \\ \\
		Αν δηλώσω ως μαθητής τον ρόλο του λογαριασμού μου, η ημερομηνία γέννησης πρέπει να είναι τέτοια ώστε να μην ξεπερνάω τα 18 έτη σε σχέση με την σημερινή. \\
		Ανάλογα το είδος του χρήστη για το οποίο αιτήθηκα θα με αποδεχτεί κατάλληλος χρήστης του συστήματος. Μετά την έγκριση εγγραφής στο σύστημα μπορώ πλέον να συνδεθώ με το \latintext{Sign In button} όπως φαίνεται παρακάτω. \\	
		\item Το \latintext{Sign In button} 
		\begin{center}
			\includegraphics[width=\textwidth]{./images/index_2.png}
		\end{center}
		Πατάμε το κουμπί όπως φαίνεται στην εικόνα και καταλήγουμε στην εξής σελίδα: \\
		\includegraphics[width=\textwidth]{./images/signin.png} \\
		Εδώ και τα 2 πεδία είναι απαραίτητα για να συνδεθούμε. \\ \\ Πατάμε το κουμπί \latintext{Sign In} και θα ενημερωθούμε για το αν βάλαμε τα στοιχεία μας λάθος ή λογαριασμός μας δεν είναι ενεργός για διάφορους λόγους(πχ. δεν έχει εγκριθεί ακόμη ή αργήσαμε να επιστρέψουμε βιβλίο). \\ \\ Αν όλα πάνε καλά, θα ανακατευθυνθούμε σε κατάλληλη αρχική σελίδα χρήστη ανάλογα το ρόλο που έχει ο λογαριασμός μας.
	\end{enumerate}

	\newpage
	\hypertarget{Buttons}{}
	\chapter*{\Large Ανάλυση κουμπιών}
	\hyperlink{student-anchor-buttons}{\Large 1)Μαθητής ή Δάσκαλος} \\
	\hyperlink{operator-anchor-buttons}{\Large 2)Χειριστής} \\ 
	\hyperlink{admin-anchor-buttons}{\Large 3)Διαχειριστής} \\
	
	\newpage
	\hypertarget{student-anchor-buttons}{}
	\chapter*{\Large Χρήστης \\(Μαθητής ή Δάσκαλος)}
	Αν ο ρόλος του λογαριασμού  μας είναι Μαθητής ή Δάσκαλος θα ανακατευθυνθούμε στην εξής σελίδα: \\
	\includegraphics[width=0.85\textwidth, height=0.85\textwidth, keepaspectratio]{./images/student_home.png}
	
	\newpage
	Αναλύουμε τα κουμπιά με τη σειρά που παρουσιάζονται: \\ \\
	\hyperlink{s-show-my-info}{(1) Show My info} \\
	\hyperlink{s-show-my-library}{(2) Show My Library} \\
	\hyperlink{s-books}{(3) Books} \\
	\hyperlink{s-past-loans}{(4) Show my Loans and Reservations} \\
	\hyperlink{s-card-status}{(5) My Card Status} \\
	\hyperlink{s-get-phones}{(6) Get my phones} \\
	\hyperlink{s-change-phones}{(7) Change my Phones} \\
	\hyperlink{s-show-loans}{(8) Show my Loans and Reservations} \\
	\hyperlink{s-show-requests}{(9) Show my requests} \\
	\hyperlink{s-delete}{(10) Delete request or reservation} \\
	\hyperlink{s-review}{(11) Add a review to a book} \\
	\hyperlink{s-change-attribute}{(12) Change My Attributes} \\
	\hyperlink{s-change-password}{(13) Change my password} \\
	\hyperlink{s-contact-library}{(14) Contact my Library}
	
	\begin{enumerate}[label=(\arabic*)]
		\newpage
		\hypertarget{s-show-my-info}{}
		\item \latintext{Show My info button} \\
		Πατώντας το κουμπί στην ίδια σελίδα μας εμφανίζονται οι πληροφορίες λογαριασμού μας: \\
		\begin{center}
			\includegraphics[width=\textwidth]{./images/student_home_with_myinfo.png}
		\end{center}
		\newpage
		\hypertarget{s-show-my-library}{}
		\item \latintext{Show My Library} \\
		 Πατώντας το κουμπί στην ίδια σελίδα μας εμφανίζονται οι πληροφορίες της βιβλιοθήκης μας: \\
		 \begin{center}
		 	\includegraphics[width=\textwidth]{./images/student_home_with_info.png}
		 \end{center}
	 	\newpage
	 	\hypertarget{s-books}{}
	 	\item \latintext{Books button} \\
	 	Πατώντας το κουμπί ανακατευθυνόμαστε στην εξής σελίδα: \\
	 	\begin{center}
	 		\includegraphics[width=\textwidth]{./images/book_search.png}
	 	\end{center}
 		Εδώ ο χρήστης μπορεί να ψάξει βιβλία που υπάρχουν στη βιβλιοθήκη χρησιμοποιώντας κριτήριο τουλάχιστον ένα από τα τίτλος, λέξεις-κλειδιά, συγγραφέας.\\
 		\newpage
 		Παρουσιάζω ως αποτέλεσμα μια αναζήτηση με κριτήριο τον τίτλο: \\
 		\includegraphics[width=\textwidth]{./images/book_search_results.png} 
 		
 		\vspace{\baselineskip}
 		Εμφανίζονται άλλα 2 κουμπιά για κάθε αποτέλεσμα αναζήτησης
 		\newpage
 		Πατώντας το summary εμφανίζεται η περίληψη του συγκεκριμένου βιβλίου: \\
 		\includegraphics[width=\textwidth]{./images/book_search_results_summary.png}
 		
 		\vspace{\baselineskip}
 		Πατώντας το κουμπί Request Book καταχωρείται αυτόματα η αίτηση του χρήστη και ανακατευθυνόμαστε στην αρχική σελίδα του απλού χρήστη. \\ \\
 		Ας σημειωθεί πως έχουμε βάλει τα ίδια όρια που υπάρχουν για τις κρατήσεις και στις αιτήσεις και γίνεται έλεγχος πριν καταχωρηθεί στο σύστημα. 
 		
 		\newpage
 		\hypertarget{s-past-loans}{}
 		\item My past Loan and Reservations \\
 		Πατώντας το κουμπί ανακατευθύνομαι στην εξής σελίδα, όπου φαίνονται τα παλιά δάνειά μου και οι παλιές μου κρατήσεις: \\
 		
 		\vspace{\baselineskip}
 		
 		\includegraphics[width=\textwidth]{./images/past_loans_reservations.png}
 		
 		\vspace{\baselineskip}
 		
 		\newpage
 		\hypertarget{s-card-status}{}
 		\item My Card Status  \\
 		Πατώντας το κουμπί ανακατευθυνόμαστε στην εξής σελίδα: \\ \\
 		\includegraphics[width=\textwidth]{./images/lost_my_card.png}
 		
 		\vspace{\baselineskip}
 		
 		Εδώ μπορούμε να δούμε την κατάσταση της κάρτας μας και να ενημερώσουμε το σύστημα πως χάσαμε την κάρτα. \\ \\
 		Ζητείται και επιβεβαίωση κωδικού για να μην μπορεί να γίνει ακούσια. \\ \\ 
 		Αφού καταχωρηθεί η απώλεια κάρτας ανακατευθυνόμαστε στην αρχική σελίδα του απλού χρήστη. \\
 		
 		\newpage
 		\hypertarget{s-get-phones}{}
 		\item Get Phones \\
 		Πατώντας το κουμπί ανακατευθυνόμαστε στην εξής σελίδα: \\
 		
 		\vspace{\baselineskip}
 		
 		\includegraphics[width=\textwidth]{./images/get_phones.png}
 		
 		\vspace{\baselineskip}
 		
 		Εδώ μπορούμε να δούμε αναλυτικά όλα τα τηλέφωνα που έχουμε δηλώσει στον λογαριασμό μας.
 		
 		\newpage
 		\hypertarget{s-change-phones}{}
 		\item Change my Phones \\
 		Πατώντας το κουμπί ανακατευθυνόμαστε στην εξής σελίδα: \\
 		
 		\vspace{\baselineskip}
 		
 		\includegraphics[width=\textwidth]{./images/change_phones.png}
 		
 		\vspace{\baselineskip}
 		
 		Εδώ εμφανίζονται 2 επιπλέον κουμπιά, το "Add Number" και το "Delete Number". \\ \\
 		\newpage
 		Για το Add Number: \\ \\ 
 		Πληκτρολογούμε τον αριθμό που θέλουμε να προσθέσουμε(προσοχή να είναι όντως 10ψήφιος) και πατάμε το κουμπί όπως φαίνεται στην παρακάτω εικόνα: \\
 		
 		\vspace{\baselineskip}
 		
 		\includegraphics[width=\textwidth]{./images/add_number.png}
 		
 		\vspace{\baselineskip}
 		Μας έρχεται ειδοποίηση αν καταχώρηθηκε σωστά ή όχι και ανανεώνεται η ίδια σελίδα. \\ \\
 		\newpage
 		Για το Delete Number: \\ \\
 		Πληκτρολογούμε το τηλέφωνο που θέλουμε να διαγράψουμε(προσοχή να είναι από τα υπάρχοντα) και πατάμε το κουμπί όπως φαίνεται στην παρακάτω εικόνα: \\ 
 		
 		\vspace{\baselineskip}
 		
 		\includegraphics[width=\textwidth]{./images/delete_number.png}
 		
 		\vspace{\baselineskip}
 		Αν δεν υπάρχει ο αριθμός απλά δεν θα διαγραφεί τίποτα. \\
 		Μας έρχεται ειδοποίηση αν καταχώρηθηκε σωστά ή όχι και ανανεώνεται η ίδια σελίδα. \\ \\
 		
 		\newpage
 		\hypertarget{s-show-loans}{}
 		\item Show my Loans and Reservations \\
 		Πατώντας το κουμπί ανακατευθυνόμαστε στην εξής σελίδα: \\
 		
 		\vspace{\baselineskip}
 		
 		\includegraphics[width=\textwidth]{./images/active_loans_reservations.png}
 		
 		\vspace{\baselineskip}
 		
 		Εδώ μπορούμε και να παρατηρήσουμε αν κάποιος δανεισμός είναι αργοπωρημένος. \\
 		Αν είναι, θα έχει στο status τον χαρακτηρισμό "Late Active".\\
 		
 		\newpage
 		\hypertarget{s-show-requests}{}
 		\item Show my requests
 		Πατώντας το κουμπί ανακατευθυνόμαστε στην εξής σελίδα: \\
 		
 		\vspace{\baselineskip}
 		
 		\includegraphics[width=\textwidth]{./images/active_requests.png}
 		
 		\vspace{\baselineskip}
 		
 		Το ID απλά απαριθμίζει τα requests
 		
 		\newpage
 		\hypertarget{s-delete}{}
 		\item Delete request or reservation \\
 		Πατώντας το κουμπί ανακατευθυνόμαστε στην εξής σελίδα: \\
 		
 		\vspace{\baselineskip}
 		
 		\includegraphics[width=\textwidth]{./images/delete_req_res.png}
 		
 		\vspace{\baselineskip}
 		
 		Για κάθε αποτέλεσμα που μας επιστρέφεται, ανεξαρτήτως αν είναι αίτηση ή κράτηση, εμφανίζονται και ένα κουμπί delete. \\
 		Πατώντας καταργείται το συγκεκριμένο αντικείμενο και ανανεώνεται η σελίδα: \\ 
 		
 		\vspace{\baselineskip}
 		
 		\includegraphics[width=\textwidth]{./images/delete_res_req_updated.png}
 		
 		\vspace{\baselineskip}
 		
 		\newpage
 		\hypertarget{s-review}{}
 		\item Add review \\ 
 		Πατώντας το κουμπί ανακατευθυνόμαστε στην εξής ιστοσελίδα: \\
 		
 		\vspace{\baselineskip}
 		
 		\includegraphics[width=\textwidth]{./images/add_review.png}
 		
 		\vspace{\baselineskip}
 		
 		Η αξιολόγηση ενός βιβλίου έχει 3 υποχρεωτικά πεδία: \\
 		α)Το ISBN του βιβλίου \\
 		β)Το likertr rating του βιβλίου \\
 		γ)Το ίδιο το κείμενο της αξιολόγησης και σχολιασμού \\ \\
 		
 		Κάθε αξιολόγηση που κάνει ο χρήστης είναι σε εκκρεμότητα μέχρι κάποιος χειριστής να την εγκρίνει για λόγους ασφάλειας και σωστής χρήσης της γλώσσας. \\ \\
 		Για να καταχωρηθεί σωστά μια αξιολόγηση από τον χρήστη πρέπει: \\
 		α)Το βιβλίο για το οποίο κάνει αξιολόγηση να υπάρχει στη βιβλιοθήκη που ανήκει \\
 		β)Να μην έχει ήδη και άλλη εκκρεμή αξιολόγηση για το βιβλίο και να μην το έχει αξιολογήσει ήδη \\
 		γ)Το likert rating να είναι ακέραιος και ανάμεσα από το 1 και το 5 \\
 		δ)Ο σχολιασμός να μην είναι κενός \\ \\
 		Σε κάθε περίπτωση ο χρήστης ενημερώνεται με κατάλληλο μήνυμα όπως φαίνεται παρακάτω και ανανεώνεται η σελίδα: \\
 		
 		\vspace{\baselineskip}
 		
 		\includegraphics[width=\textwidth]{./images/add_review_success.png}
 		
 		\vspace{\baselineskip}
 		
 		\vspace{\baselineskip}
 		
 		\includegraphics[width=\textwidth]{./images/add_review_reject.png}
 		
 		\vspace{\baselineskip}
 		
 		\newpage
 		\hypertarget{s-change-attribute}{}
 		\item Change My Attributes(Αποκλειστικά για δασκάλους) \\
 		Παρακάτω φαίνεται το επιπλέον κουμπί: \\
 		
 		\vspace{\baselineskip}
 		
 		\includegraphics[width=\textwidth]{./images/change_attributes_button.png}
 		
 		\vspace{\baselineskip}
 		
 		\newpage
 		Πατώντας το κουμπί ανακατευθυνόμαστε στην εξής σελίδα: \\
 		
 		\vspace{\baselineskip}
 		
 		\includegraphics[width=\textwidth]{./images/change_attributes.png}
 		
 		\vspace{\baselineskip}
 		
 		Για να μπορέσουμε να υποβάλλουμε αλλαγή πρέπει να συμπληρώσουμε όλα τα πεδία(μας ειδοποιεί το σύστημα αν δεν το κάνουμε), πατώντας submit αν όλα πάνε καλά θα ανακατευθυνθούμε στην αρχική σελίδα του δασκάλου, αλλιώς θα ενημερωθούμε κατάλληλα. \\
 		
 		\newpage
 		\hypertarget{s-change-password}{}
 		\item Change my password \\
 		Πατώντας το κουμπί ανακατευθυνόμαστε στην εξής σελίδα: \\
 		
 		\vspace{\baselineskip}
 		
 		\includegraphics[width=\textwidth]{./images/change_password.png}
 		
 		\vspace{\baselineskip}
 		
 		Για να γίνει σωστή αλλαγή πρέπει να συμπληρωθούν και τα 3 πεδία μαζί. \\
 		Το σύστημα ελέγχει αν ο παλιός κωδικός είναι σωστός και αν οι καινούριοι κωδικοί δεν είναι διαφορετικοί. \\
 		Στην περίπτωση επιτυχούς αλλαγής ανακατευθυνόμαστε στην αρχικήκ σελίδα του χρήστη. 
 		
 		\newpage
 		\hypertarget{s-contact-library}{}
 		\item Contact my Library \\
 		Πατώντας το κουμπί ανακατευθυνόμαστε στην εξής σελίδα: \\
 		
 		\vspace{\baselineskip}
 		
 		\includegraphics[width=\textwidth]{./images/contact_my_library.png}
 		
 		\vspace{\baselineskip}
 		
 		Εδώ φαίνονται τα στοιχεία επικοινωνίας της βιβλιοθήκης που ανήκω. \\
	\end{enumerate}
	
	\newpage
	\hypertarget{operator-anchor-buttons}{}
	\chapter*{\Large Χειριστής}
	Αν ο ρόλος του λογαριασμού  μας είναι Χειριστής θα ανακατευθυνθούμε στην εξής σελίδα: \\
	\includegraphics[width=\textwidth, height=\textwidth, keepaspectratio]{./images/operator_home.png}
	
	\vspace{\baselineskip}
	
	Ο χειριστής έχει όλες τις λειτουργίες του μαθητή και επιπλέον ένα κουμπί που λέγεται query my database καθώς και ένα ολόκληρο επιπλέον πίνακα με κουμπιά για τη διαχείριση της βιβλιοθήκης του. \\ \\
	
	\newpage
	Αναλύουμε τα κουμπιά με τη σειρά που παρουσιάζονται στον επιπλέον πίνακα με τα κουμπιά: \\ \\
	\hyperlink{o-show-pending-reg}{(1) Show Pending Registrations} \\
	\hyperlink{o-pending-cards}{(2) Pending Cards} \\
	\hyperlink{o-manage-users}{(3) Manage Users} \\
	\hyperlink{o-manage-library}{(4) Manage Library} \\
	\hyperlink{o-manage-request}{(5) Manage Requests} \\
	\hyperlink{o-late-returns}{(6) Late Returns} \\
	\hyperlink{o-satisfy-res}{(7) Satisfy Reservations} \\
	\hyperlink{o-instant-loans}{(8) Instant Loans} \\
	\hyperlink{o-return-book}{(9) Return book of a student} \\
	\hyperlink{o-get-active-loans}{(10) Get active loans} \\
	\hyperlink{o-change-book}{(11) Change Existing Book} \\
	\hyperlink{o-manage-pending-reviews}{(12) Manage Pending Reviews} \\
	\hyperlink{o-new-book}{(13) New Book!!!} \\
	\hyperlink{o-query-database}{(14) Query my Database} \\
	
	
	\begin{enumerate}
		\newpage
		\hypertarget{o-show-pending-reg}{}
		\item Show Pending Registrations \\
		Πατώντας το συγκεκριμένο κουμπί ανακατευθυνόμαστε στην εξής σελίδα: \\
		\vspace{\baselineskip}
		
		\includegraphics[width=0.8\textwidth]{./images/pending_registrations.png}
		
		\vspace{\baselineskip}
		
		\newpage
		Παρατηρούμε πως για κάθε αποτέλεσμα, εκκρεμή εγγραφή, έχω 2 επιπλέον κουμπιά: Accept, Deny \\ \\
		Πατώντας ένα από αυτά όπως φαίνεται παρακάτω καταχωρείται η αντίστοιχη πράξη στο σύστημα και ανανεώνεται η σελίδα: \\
		
		\vspace{\baselineskip}
		
		\includegraphics[width=0.8\textwidth]{./images/pending_registrations_accept.png}
		
		\vspace{\baselineskip}
		
		\newpage
		\hypertarget{o-pending-cards}{}
		\item Pending Cards \\
		Πατώντας το κουμπί ανακατευθυνόμαστε στην εξής σελίδα: \\
		
		
		\vspace{\baselineskip}
		
		\includegraphics[width=\textwidth]{./images/pending_cards.png}
		
		\vspace{\baselineskip}
		
		Ακριβώς η ίδια λογική με το κουμπί Show Pending Registrations, δηλαδή έχω επιπλέον 2 κουμπιά: accept, deny \\
		Πατώντας ένα από αυτά όπως φαίνεται παρακάτω καταχωρείται η αντίστοιχη πράξη στο σύστημα και ανανεώνεται η σελίδα. \\
		
		\newpage
		\hypertarget{o-manage-users}{}
		\item Manage Users \\
		Πατώντας το κουμπί ανακατευθυνόμαστε στην εξής σελίδα: \\
		
		\vspace{\baselineskip}
		
		\includegraphics[width=\textwidth]{./images/manage_users_empty.png}
		
		\vspace{\baselineskip}
		
		Αναζητούμε χρήστες με το επίθετο και βλέπουμε τα εξείς αποτελέσματα: \\
		
		\vspace{\baselineskip}
		
		\includegraphics[width=\textwidth]{./images/manage_users.png}
		
		\vspace{\baselineskip}
		
		Παρατηρούμε πως για τον μοναδικό χρήστη που μας εμφαζίνεται έχουμε 3 συγκεκριμένα κουμπιά: Activate, Deactivate, Delete. \\ \\
		α)Deactivate μπορεί να χρειαστεί να κάνουμε γιατί για παράδειγμα ο χρήστης καθυστέρησε την επιστροφή βιβλίου ή έκανε κάποια ζημιά ως ποινή. \\
		β)Delete μπορεί να χρειαστεί να κάνουμε γιατί για παράδειγμα ο χρήστης έφυγε από το σχολείο. \\
		γ)Activate μπορεί να χρειαστεί να κάνουμε για να αναιρέσουμε οποιαδήποτε από τις 2 παραπάνω ενέργειες. \\ \\
		Πατώντας οποιαδήποτε από τα 3 πεδία ενημερώνεται ο χειριστής κατάλληλα και ανανεώνεται η σελίδα όπως φαίνεται παρακάτω: \\
		
		\vspace{\baselineskip}
		
		\includegraphics[width=\textwidth]{./images/manage_users_deactivation.png}
		
		\vspace{\baselineskip}
		
		\newpage
		\hypertarget{o-manage-library}{}
		\item Manage Library \\
		Πατώντας το κουμπί ανακατευθυνόμαστε στην εξής σελίδα: \\
		
		\vspace{\baselineskip}
		
		\includegraphics[width=\textwidth]{./images/manage_library.png}
		
		\vspace{\baselineskip}
		
		Μπορούμε να επιτελέσουμε 2 λειτουργίες: Αναζήτηση Βιβλίου στο σύστημα, Εισαγωγική Βιβλίου στη βιβλιοθήκη μου
		\newpage
		α) Αναζητώντας βιβλία με βάση τον τίτλο και πατώντας το Search button παίρνουμε αποτελέσματα που εμφανίζονται όπως παρακάτω: \\
		
		\vspace{\baselineskip}
		
		\includegraphics[width=\textwidth]{./images/manage_library_search.png}
		
		\vspace{\baselineskip}
		
		\newpage
		β) Εισάγω βιβλία στην βιβλιοθήκη μου όπως φαίνεται παρακάτω: \\
		
		\vspace{\baselineskip}
		
		\includegraphics[width=\textwidth]{./images/manage_library_insert.png}
		
		\vspace{\baselineskip}
		
		Μετά την εισαγωγή ανανεώνεται η σελίδα.
		
		\newpage
		\hypertarget{o-manage-request}{}
		\item Manage request \\
		Πατώντας το κουμπί ανακατευθυνόμαστε στην εξής σελίδα: \\
		
		\vspace{\baselineskip}
		
		\includegraphics[width=\textwidth]{./images/manage_request.png}
		
		\vspace{\baselineskip}
		
		Για κάθε αποτέλεσμα έχουμε 2 κουμπιά: Accept, Delete \\
		Κάθε κουμπί επιτελεί την προφανή λειτουργία, αφαιρείται η αίτηση από τη λίστα που παρουσιάζεται και ανανεώεται η σελίδα \\
		
		\newpage
		\hypertarget{o-late-returns}{}
		\item Late Returns \\
		Πατώντας το κουμπί ανακατευθυνόμαστε στην εξής σελίδα: \\
		
		\vspace{\baselineskip}
		
		\includegraphics[width=0.7\textwidth, height=0.7\textheight, keepaspectratio]{./images/late_returns.png}
		
		\vspace{\baselineskip}
		
		Μας παρουσιάζει αναλυτικά η "μαύρη" λίστα, με όσους έχουν καθυστερημένες επιστροφές βιβλίων.
		
		\newpage
		\hypertarget{o-satisfy-res}{}
		\item Satisfy Reservations \\
		Πατώντας το κουμπί ανακατευθυνόμαστε στην εξής σελίδα: \\
		
		\vspace{\baselineskip}
		
		\includegraphics[width=\textwidth]{./images/satisfy_reservations.png}
		
		\vspace{\baselineskip}
		
		Για κάθε αποτέλεσμα έχουμε 2 κουμπιά: Honour, Move to Expired \\
		Με το honour ικανοποιούμε την κράτηση και φτιάχνουμε καινούριο δανεισμό. \\
		Με το Move to Expired δεν ικανοποιούμε την κράτηση και πρακτικά την απορρίπτουμε βάζοντάς της στα Expired. \\
		
		\newpage
		\hypertarget{o-instant-loans}{}
		\item Instant Loans \\
		Πατώντας το κουμπί ανακατευθυνόμαστε στην εξής σελίδα: \\
		
		\vspace{\baselineskip}
		
		\includegraphics[width=\textwidth]{./images/instant_loans.png}
		
		\vspace{\baselineskip}
		
		\newpage
		Συμπληρώνουμε και τα 2 πεδία και πατάμε submit όπως φαίνεται παρακάτω: \\
		
		\vspace{\baselineskip}
		
		\includegraphics[width=\textwidth]{./images/instant_loans_submit.png}
		
		\vspace{\baselineskip}
		
		Για να μπορέσει να ικανοποιηθεί το αίτημα πρέπει: \\ \\
		α)Ο χρήστης στον οποίο καταχωρούμε τον δανεισμό να μην έχει καθυστερημένες επιστροφές βιβλίου \\
		β)Ο χρήστης στον οποίο καταχωρούμε τον δανεισμό να έχει περιθώριο στον αριθμό βιβλίων που δανείζεται ανά βδομάδα, ανάλογα τον ρόλο του \\
		γ)Το βιβλίο να υπάρχει στη βιβλιοθήκη μου \\
		δ)Να υπάρχει διαθέσιμο αντίτυπο στην βιβλιθήκη μου \\ \\
		Σε κάθε περίπτωση θα ενημερωθώ κατάλληλα με μήνυμα και θα ανανεωθεί η σελίδα. \\
		
		\newpage
		\hypertarget{o-return-book}{}
		\item Return book of a student \\
		Πατώντας το κουμπί ανακατευθυνόμαστε στην εξής σελίδα: \\
		
		\vspace{\baselineskip}
		
		\includegraphics[width=\textwidth]{./images/return_book.png}
		
		\vspace{\baselineskip}
		
		\newpage
		Συμπληρώνουμε και τα 2 πεδία και πατάμε submit όπως φαίνεται παρακάτω: \\
		
		\vspace{\baselineskip}
		
		\includegraphics[width=\textwidth]{./images/return_book_submit.png}
		
		\vspace{\baselineskip}
		
		Η μόνη προϋπόθεση για να καταχωριστεί σωστά η επιστροφή του βιβλίου είναι να μην το έχει επιστρέψει ήδη. \\
		Ανάλογα με την ημερομηνία δανεισμού, το σύστημα καταχωρεί αν ήταν κανονική επιστροφή ή καθυστερημένει και ενημερώνει τον χρήστη ότι η επιστροφή καταχωρήθηκε. Τέλος ανανεώνεται η σελίδα. \\
		
		\newpage
		\hypertarget{o-get-active-loans}{}
		\item Get active loans \\
		Πατώνας το κουμπί ανακατευθυνόμαστε στην εξής σελίδα: \\
		
		\vspace{\baselineskip}
		
		\includegraphics[width=\textwidth]{./images/active_loans.png}
		
		\vspace{\baselineskip}
		
		Παρουσιάζονται αναλυτικά όλοι οι ενεργοί δανεισμοί. \\
		
		\newpage
		\hypertarget{o-change-book}{}
		\item  Change Existing Book \\
		Πατώντας το κουμπί ανακατευθυνόμαστε στην εξής σελίδα: \\
		
		\vspace{\baselineskip}
		
		\includegraphics[width=\textwidth]{./images/change_existing_book_2.png}
		
		\vspace{\baselineskip}
		
		\newpage
		Έχουμε 2 λειτουργίες: Διαγραφή προϋπάρχοντος βιβλίου, Αλλαγή προϋπάροχντος βιβλίου. \\ \\
		α)Διαγραφή προϋπάρχοντος βιβλίου: εισάγουμε το ISBN του βιβλίου και πατάμε το κουμπί Delete from Database.\\
		Διαγράφεται το βιβλίο και ανανεώνεται η σελίδα. \\ 
		\newpage
		β) Εισαγωγή καινούριου βιβλίου: \\
		
		\vspace{\baselineskip}
		
		\includegraphics[width=\textwidth]{./images/change_existing_book_1.png}
		
		\vspace{\baselineskip}
		
		Για να αλλάξουμε ένα προϋπάρχον βιβλίο πρέπει να συμπληρώσω όλα τα πεδία. \\
		Σε περίπτωση που έχουμε κάνει λάθος το ISBN ή η φωτογραφία μας δεν υποστήριζεται ή η φωτογραφία μας είναι πολύ μεγάλη το σύστημα μας ενημερώνει κατάλληλα. \\
		Σε περίπτωση που όλα πάνε καλά ανανεώνεται η σελίδα. \\
		
		\newpage
		\hypertarget{o-manage-pending-reviews}{}
		\item Manage Pending Reviews \\
		Πατώντας το κουμπί ανακατευθυνόμαστε στην εξής σελίδα: \\
		
		\vspace{\baselineskip}
		
		\includegraphics[width=\textwidth]{./images/manage_pending_reviews.png}
		
		\vspace{\baselineskip}
		
		Για κάθε αποτέλεσμα έχουμε 2 κουμπιά: Accept, Delete \\
		Κάθε κουμπί εκτελεί την λειτουργία του και ανανεώνεται η σελίδα. \\
		Χρειάζεται να περάσουμε από αυτό το στάδιο γιατί όλα τα στατιστικά της βάσης προκύπτουν από τις αξιολογήσεις που έχουμε αποδεκτοί και όχι από τις εκκρεμείς αξιολογήσεις που μπορεί να κάνει ο απλός χρήστης-μαθητής. \\
		
		\newpage
		\hypertarget{o-new-book}{}
		\item New Book!!! (Μεγάλο κουμπί κάτω κάτω) \\
		Πατώντας το κουμπί ανακατευθυνόμαστε στην εξής σελίδα: \\
		
		\vspace{\baselineskip}
		
		\includegraphics[width=\textwidth, height=0.7\textheight, keepaspectratio]{./images/new_book.png}
		
		\vspace{\baselineskip}
		
		Πρέπει να συμπληρώσουμε όλα τα πεδία για να μπορέσουμε να καταχωρήσουμε ένα καινούριο βιβλίο στο σύστημα. \\
		Άμα καταχωρηθεί επιτυχώς ανακατευθυνόμαστε στην αρχική σελίδα του χειριστή αλλιώς θα ειδοποιηθούμε με σφάλμα. 
		
		\newpage
		\hypertarget{o-query-database}{}
		\item Query my Database \\
		Πατώντας το κουμπί ανακατευθυνόμαστε στην εξής σελίδα: \\

		\vspace{\baselineskip}
		
		\includegraphics[width=0.85\textwidth]{./images/operator_query.png}
		
		\vspace{\baselineskip}
		
		Σ' αυτή τη σελίδα υπάρχουν 3 κοινά κουμπιά με τον μαθητή απλά μεταφέρθηκαν εδώ γιατί θεωρήσαμε πως είναι ερωτήσεις προς τη βάση και είναι σχετικές με το θέμα. \\
		Πιο συγκεκριμένα τα 3 κουμπιά είναι: \\
		\hyperlink{s-get-phones}{α)Get my phones} \\
		\hyperlink{s-show-loans}{β)Show my Loans and Reservations} \\
		\hyperlink{s-show-requests}{γ)Show my requests} \\ \\
		\newpage
		Τα υπόλοιπα 5 κουμπιά αναλύονται παρακάτω: \\ \\
		\hyperlink{o-show-info-lib}{ (α) Show My Info/Library} \\
		\hyperlink{o-book-search}{(β) Book Search} \\
		\hyperlink{o-delayed-loans}{(γ) Delayed Loan Search} \\
		\hyperlink{o-loan-rating}{(δ) Loan Rating Search} 
		
		\begin{enumerate}
			\newpage
			\hypertarget{o-show-info-lib}{}
			\item Show My Info/Show My Library \\
			Πατώντας τα κουμπιά εμφανίζονται επιπλέον πληροφορίες για το λογαριασμό μου και τη βιβλιοθήκη μου: \\
			\vspace{\baselineskip}
			
			\includegraphics[width=\textwidth]{./images/query_operator_show_stuff.png}
			
			\vspace{\baselineskip}
			
			\newpage
			\hypertarget{o-book-search}{}
			\item Book Search \\
			Πατώντας το κουμπί ανακατευθυνόμαστε στην εξής σελίδα: \\
			
			\vspace{\baselineskip}
			
			\includegraphics[width=\textwidth]{./images/query_book_search.png}
			
			\vspace{\baselineskip}
			
			Εδώ μπορούμε να αναζητήσουμε βιβλία με βάση: \\
			1)Τίτλο \\
			2)Κατηγορία \\
			3)Όνομα συγγραφέα \\
			4)Ελάχιστα διαθέσιμα αντίτυπα \\ \\
			Στο παράδειγμα που φαίνεται στην φωτογραφία από πάνω έκανα αναζήτηση με ελάχιστο αριθμό 10 αντίτυπα διαθέσιμα
			
			\newpage 
			\hypertarget{o-delayed-loans}{}
			\item Delayed Loan Search \\
			Πατώντας το κουμπί ανακατευθυνόμαστε στην εξής σελίδα: \\
			
			\vspace{\baselineskip}
			
			\includegraphics[width=\textwidth]{./images/delayed_loan.png}
			
			\vspace{\baselineskip}
			
			Υπάρχουν 3 επιλογές για την εύρεση ατόμων που έχουν βιβλίο και δεν το έχουν επιστρέψει ακόμη παρ'όλο που έχει περάσει 1 βδομάδα: \\
			α)Όνομα \\
			β)Επίθετο \\
			γ)Μέρες πέραν του ορίου \\ \\
			\newpage
			Παρατίθεται παράδειγμα με ονοματεπώνυμο: \\
			
			\vspace{\baselineskip}
			
			\includegraphics[width=\textwidth]{./images/delayed_loan_results.png}
			
			\vspace{\baselineskip}
			
			\newpage
			\hypertarget{o-loan-rating}{}
			\item Loan Rating Search \\
			Πατώντας το κουμπί ανακατευθυνόμαστε στην εξής σελίδα: \\
			
			\vspace{\baselineskip}
			
			\includegraphics[width=\textwidth]{./images/loan_rating.png}
			
			\vspace{\baselineskip}
			
			Υπάρχουν 2 επιλογές αναζήτησης για την εύρεση μέσο όρου αξιολογήσεων που έχουν εγκριθεί από χειριστή: \\
			α)UserID \\
			β)Κατηγορία βιβλίου \\ \\
			Καθένα από τα παραπάνω μπορεί να χρησιμοποιηθεί είτε ξεχωριστά είτε συνδυασμός και των 2. \\ \\ 
			\newpage
			α)Μόνο χρήστης \\ 
			\vspace{\baselineskip}
			
			\includegraphics[width=\textwidth]{./images/loan_rating_user.png}
			
			\vspace{\baselineskip}
			
			\newpage
			β)Μόνο κατηγορία \\
			\vspace{\baselineskip}
			
			\includegraphics[width=\textwidth]{./images/loan_rating_category.png}
			
			\vspace{\baselineskip}
			
			\newpage
			γ)Χρήστης κα κατηγορία μαζί \\
			\vspace{\baselineskip}
			
			\includegraphics[width=\textwidth]{./images/loan_rating_user_and_category.png}
			
			\vspace{\baselineskip}
		\end{enumerate}
	\end{enumerate}

	\newpage
	\hypertarget{admin-anchor-buttons}{}
	\chapter*{\Large Διαχειριστής} 
	Ο διαχειριστής είναι υπεύθυνος για τις πιο σημαντικές λειτουργίες της βάσης μας. Δεν ασχολείται τοπικά με ξεχωριστές βιβλιοθήκες όπως ένας χειριστής αλλά συνολικά με το σύστημα. \\ Παρ 'όλα αυτά έχουμε κάνει την παραδοχή πως μπορεί να έχει και τις δυνατότητες του απλού χρήστη και να μπορεί να δανείζεται και αυτός βιβλία από βιβλιοθήκες. Επιπλέον έχουμε κάνει την παραδοχή πως μπορεί να δανειστεί απεριόριστα βιβλία δημοκρατικά.
	
	\newpage
	Όταν ο ρόλος του λογαριασμού μας είναι 'Διαχειριστής' ανακατευθυνόμαστε μετά το sign in στην παρακάτω σελίδα: \\
	
	\vspace{\baselineskip}
	
	\includegraphics[width=0.85\textwidth]{./images/admin_home.png}
	
	\vspace{\baselineskip}
	
	Όπως προαναφέρθηκε ο διαχειριστής μπορεί να επιτελέσει και όλες τις λειτουργίες του απλού χρήστη. \\
	Άρα το μόνο επιπλέον κουμπί στη ΡΟΖ ομάδα κουμπιών είναι το query my database, το οποίο θα αναλυθεί τελευταίο. \\ 
	
	\newpage
	Παρακάτω παρατίθονται αναλυτικά οι λειτουργίες του εκάτοστε κουμπιού: \\ \\
	\hyperlink{a-operator-registrations}{(1) Operator-Admin Registrations} \\
	\hyperlink{a-add-change-library}{(2) Add or Change Library} \\
	\hyperlink{a-create-backup}{(3) Create Backup of Database} \\
	\hyperlink{a-restore-database}{(4) Restore Database from Backup} \\
	\hyperlink{a-query-database}{(5) Query my database} \\
	
	\begin{enumerate}
		\newpage
		\hypertarget{a-operator-registrations}{}
		\item Operator-Admin Registrations \\
		Πατώντας το παραπάνω κουμπί ανακατευθυνόμαστε στην εξής σελίδα: \\
		
		\vspace{\baselineskip}
		
		\includegraphics[width=\textwidth]{./images/admin_operator_registrations.png}
		
		\vspace{\baselineskip}
		
		Για κάθε ανοιχτή αίτηση χειριστή ή διαχειριστή, υπάρχουν 2 κουμπιά: Accept, Deny. \\
		Πατώντας ένα από τα 2 επιτελεί την προφανή λειτουργία και ανακατευθυνόμαστε στην αρχική σελίδα του διαχειριστή. \\ \\
		Για να επιτευχθεί σωστά έγκριση αίτησης χειριστή βιβλιοθήκης πρέπει να ΜΗΝ υπάρχει ήδη χειριστής στη βιβλιοθήκη του αιτούμενου. \\
		
		\newpage
		\hypertarget{a-add-change-library}{}
		\item Add or Change Library \\
		Πατώντας το κουμπί ανακατευθυνόμαστε στην εξής σελίδα: \\
		
		\vspace{\baselineskip}
		
		\includegraphics[width=\textwidth]{./images/admin_add_change_library.png}
		
		\vspace{\baselineskip}
		
		Υπάρχουν 2 νέα κουμπιά ανάλογα αν θέλουμε να προσθέσουμε ή να αλλάξουμε μια ήδη υπάρχουσα βιβλιοθήκη. \\
		
		\newpage
		α)Create Library \\
		Πατώντας το κουμπί ανακατευθυνόμαστε στην παρακάτω σελίδα: \\
		
		\vspace{\baselineskip}
		
		\includegraphics[width=\textwidth]{./images/admin_add_library.png}
		
		\vspace{\baselineskip}
		
		Απαραίτητα πεδία για την καταχώρηση καινούριας βιβλιοθήκης είναι όλα ΕΚΤΟΣ του τηλεφώνου. \\ \\
		Πατώντας Create καταχωρείται επιτυχώς η βιβλιοθήκη στο σύστημα αν το όνομα της βιβλιοθήκης δεν υπάρχει ήδη.
		
		\newpage
		β)Change Library Attributes \\
		Πατώντας το κουμπί ανακατευθυνόμαστε στην εξής σελίδα: \\
		
		
		\vspace{\baselineskip}
		
		\includegraphics[width=0.8\textwidth]{./images/admin_change_library.png}
		
		\vspace{\baselineskip}
		
		Χρειάζεται να συμπληρώσουμε όλα τα πρώτα 5 πεδία για να μπορέσουμε να καταχωρήσουμε επιτυχώς μια αλλαγή στα στοιχεία μια βιβλιοθήκης. \\
		Αν η αλλαγή καταχωρηθεί επιτυχώς ανακατευθυνόμαστε στην αρχική σελίδα του διαχειριστή. 
		
		\newpage
		\hypertarget{a-create-backup}{}
		\item Create-Backup of Database \\
		Πατώντας το κουμπί ανακατευθυνόμαστε στην εξής σελίδα: \\
		
		\vspace{\baselineskip}
		
		\includegraphics[width=\textwidth]{./images/admin_create_backup.png}
		
		\vspace{\baselineskip}
		
		Πατώντας το κουμπί  Create Backup φτιάχνεται ένας φάκελο που έχει όνομα './backup/dbbackup/current-date-time', \\
		όπου current-date-time η ώρα και η μέρα την στιγμή που πατήθηκε το κουμπί σε format YYYY-MM-DD-HH-MM-SS. \\
		
		Ενημερωνόμαστε κατάλληλα αν ήταν επιτυχές το backup: \\
		
		\vspace{\baselineskip}
		
		\includegraphics[width=\textwidth]{./images/admin_create_backup_alert.png}
		
		\vspace{\baselineskip}
		
		\newpage
		\hypertarget{a-restore-database}{}
		\item Restore Database From Backup \\
		Πατώντας το κουμπί ανακατευθυνόμαστε στην εξής σελίδα: \\
		
		\vspace{\baselineskip}
		
		\includegraphics[width=\textwidth]{./images/admin_restore_database.png}
		
		\vspace{\baselineskip}
		
		Επιλέγουμε το directory στο οποίο είναι αποθηκευμένο το backup της βάσης δεδομένων που είχαμε δημιουργήσει με το κουμπί Create Backup. \\
		Πατώντας το restore θα δημιουργηθεί μια βάση semester project backup που θα περιέχει όλα τα δεδομένα της βάσης του επιλεγμένου από το χρήστη backup. \\
		
		\newpage
		\hypertarget{a-query-database}{}
		\item Query my database \\
		Πατώντας το κουμπί ανακατευθυνόμαστε στην εξής σελίδα: \\
		
		\vspace{\baselineskip}
		
		\includegraphics[width=0.9\textwidth]{./images/admin_query_home.png}
		
		\vspace{\baselineskip}
		
		Τα πρώτα 4 κουμπιά είναι κοινά με το χρήστη και αναλυόνται στα πεδία του Μαθητή: \\
		\hyperlink{s-books}{(1)Books} \\ 
		\hyperlink{s-get-phones}{(2) Get my phones} \\ 
		\hyperlink{s-show-loans}{(3)Show my loans and reservations} \\
		\hyperlink{s-show-requests}{(4) Show my request} 
		\newpage
		Παρακάτω παρατίθονται αναλυτικά του υπόλοιπα κουμπιά του συγκεκριμένου query home: \\ \\
		\hyperlink{a-library-stats}{(α) Library Statistics} \\
		\hyperlink{a-author-stats}{(β) Author Statistics} \\
		\hyperlink{a-teacher-stats}{(γ) Teachers Statistics} \\
		\hyperlink{a-top-category-pairs}{(δ) Top Category Pairs} \\
		\hyperlink{a-authors-not-borrowed}{(ε) Authors Not Borrowed} \\
		\hyperlink{a-operator-loan-count}{(στ) Operators Loan Count} \\
		\hyperlink{a-authors-less-books}{(ζ) Authors which have 5(or more) less books loaned than the top author} \\
		
		\newpage
		\begin{enumerate}
			\hypertarget{a-library-stats}{}
			\item Library Statistics \\
			Πατώντας το κουμπί ανακατευθυνόμαστε στην εξής σελίδα: \\
			
			\vspace{\baselineskip}
			
			\includegraphics[width=\textwidth]{./images/admin_library_statistics.png}
			
			\vspace{\baselineskip}
			
			Υπάρχουν 3 επιλογές για να δούμε πόσους δανεισμούς έχει κάνει κάθε βιβλιοθήκη: \\ \\
			\newpage
			1)Ανά χρόνο: \\
			
			\vspace{\baselineskip}
			
			\includegraphics[width=\textwidth]{./images/admin_library_statistics_results1.png}
			
			\newpage
			\vspace{\baselineskip}
			2)Ανά μήνα: \\
			
			\vspace{\baselineskip}
			
			\includegraphics[width=\textwidth]{./images/admin_library_statistics_results2.png}
			
			\vspace{\baselineskip}
			3)Ανά μήνα και χρόνο: \\	
			Στη συγκεκριμένη περίπτωση όπως έχουμε βάλει τους δανεισμούς στη βάση δεν έχει διαφορά με το Ανά μήνα. \\
			
			\newpage
			\hypertarget{a-author-stats}{}
			\item Author Statistics \\
			Πατώντας το κουμπί ανακατευθυνόμαστε στην εξής σελίδα: \\
			
			\vspace{\baselineskip}
			
			\includegraphics[width=\textwidth]{./images/admin_author_statistics.png}
			
			\vspace{\baselineskip}
			
			Επιλέγουμε από τη μπάρα τη κατηγορία που μας ενδιαφέρει και πατάμε Get Authors. \\
			Παρακάτω φαίνεται παράδειγμα αποτελεσμάτων: \\
			
			\vspace{\baselineskip}
			
			\includegraphics[width=\textwidth]{./images/admin_author_statistics_results.png}
			
			\vspace{\baselineskip}
			
			\newpage
			\hypertarget{a-teacher-stats}{}
			\item Teacher Statistics \\
			Πατώντας το κουμπί ανακατευθυνόμαστε στην εξής σελίδα: \\
			
			\vspace{\baselineskip}
			
			\includegraphics[width=\textwidth]{./images/admin_teacher_statistics.png}
			
			\vspace{\baselineskip}
			
			Πατώντας το κουμπί Get Statistics λαμβάνουμε τους εκπαιδευτικούς που είναι μικρότεροι των 40, έχουν δανειστεί τουλάχιστον ένα βιβλίο και είναι ταξινομημένοι ως προς τον αριθμό που δανείστηκαν. Παρακάτω φαίνεται το αποτέλεσμα: \\
			
			\vspace{\baselineskip}
			
			\includegraphics[width=\textwidth]{./images/admin_teacher_statistics_results.png}
			
			\vspace{\baselineskip}
			
			\newpage
			\hypertarget{a-top-category-pairs}{}
			\item Top Category Pairs \\
			Πατώντας το κουμπί ανακατευθυνόμαστε στην εξής σελίδα: \\
			
			\vspace{\baselineskip}
			
			\includegraphics[width=\textwidth]{./images/admin_top_category_pairs.png}
			
			\vspace{\baselineskip}
			
			Πατώντας το κουμπί Get Top Pairs λαμβάνουμε τους 3 καλύτερους συνδυασμούς κατηγοριών βιβλίων σε δανεισμούς. Παρακάτω φαίνεται παράδειγμα αποτελέσματος: \\
			
			\vspace{\baselineskip}
			
			\includegraphics[width=\textwidth]{./images/admin_top_category_pair_results.png}
			
			\vspace{\baselineskip}
			
			\newpage
			\hypertarget{a-authors-not-borrowed}{}
			\item  Authors Not Borrowed \\
			Πατώντας το κουμπί ανακατευθυνόμαστε στην εξής σελίδα: \\
			
			\vspace{\baselineskip}
			
			\includegraphics[width=\textwidth]{./images/admin_authors_not_borrowed.png}
			
			\vspace{\baselineskip}
			
			Πατώντας το κουμπί Get Authors, όπως κάναμε και σε όλα τα παραπάνω κουμπιά λαμβάνουμε τους συγγραφείς που έχουν γράψει τουλάχιστον ένα βιβλίο στη βάση και κανείς δεν έχει δανειστεί βιβλίο τους. Το αποτέλεσμα φαίνεται στην παραπάνω εικόνα.
			
			\newpage
			\hypertarget{a-operator-loan-count}{}
			\item Operator Loan Count \\
			Πατώντας το κουμπί ανακατευθυνόμαστε στην εξής σελίδα: \\
			
			\vspace{\baselineskip}
			
			\includegraphics[width=\textwidth]{./images/admin_operators_loan_count.png}
			
			\vspace{\baselineskip}
			
			Πατώντας το κουμπί Get Operators μας εμφανίζονται ποιοι χειριστές έχουν δανείσει τον ίδιο αριθμό βιβλίων, με τουλάχιστον 20 βιβλία, σε ένα έτος. Το αποτέλεσμα φαίνεται παρακάτω: \\
			
			\vspace{\baselineskip}
			
			\includegraphics[width=\textwidth]{./images/operator_loan_count.png}
			
			\vspace{\baselineskip}
			
			\newpage
			\hypertarget{a-authors-less-books}{}
			\item Authors which have 5 (or more) less books loaned than the top author \\
			Πατώντας το κουμπί ανακατευθυνόμαστε στην εξής σελίδα: \\
			
			\vspace{\baselineskip}
			
			\includegraphics[width=\textwidth]{./images/admin_authors_with_less_books.png}
			
			\vspace{\baselineskip}
			
			\newpage
			Πατώντας το κουμπί Get Authors, όπως κάναμε και σε όλα τα παραπάνω κουμπιά λαμβάνουμε τους συγγραφείς που έχουν γράψει τουλάχιστον 5 βιβλία λιγότερα από τον
			συγγραφέα με τα περισσότερα βιβλία. Το αποτέλεσμα φαίνεται στην παραπάνω εικόνα.
			
		\end{enumerate}
	\end{enumerate}
\end{document}